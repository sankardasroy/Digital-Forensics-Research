%
% It is an example file showing how to use the 'acm_proc_article-sp.cls' V3.2SP
% LaTeX2e document class file for Conference Proceedings submissions.

\documentclass{mcurcsm}
%\documentclass{acm_proc_article-sp}

\usepackage[letterpaper,breaklinks=true]{hyperref}
\usepackage{breakurl}
\usepackage{balance}    % to balance the last page's columns

\usepackage{graphicx}
\usepackage{subcaption}

\usepackage{color}
\newcommand{\comment}[1]{{\color{red}{NOTE: {#1}}}}

\begin{document}

    \title{NIST Standards Compliance of Metadata-Based Deleted-File-Recovery (DFR) Tools}
%   \subtitle{[And a sample article]
%   \titlenote{A full version of this paper is available as
%   \textit{Author's Guide to Preparing ACM SIG Proceedings Using
%   \LaTeX$2_\epsilon$\ and BibTeX} at
%   \url{http://www.acm.org/eaddress.htm}}}

%
% You need the command \numberofauthors to handle the 'placement
% and alignment' of the authors beneath the title.
%
% For aesthetic reasons, we recommend 'three authors at a time'
% i.e. three 'name/affiliation blocks' be placed beneath the title.
%
% NOTE: You are NOT restricted in how many 'rows' of
% "name/affiliations" may appear. We just ask that you restrict
% the number of 'columns' to three.
%
% Because of the available 'opening page real-estate'
% we ask you to refrain from putting more than six authors
% (two rows with three columns) beneath the article title.
% More than six makes the first-page appear very cluttered indeed.
%
% Use the \alignauthor commands to handle the names
% and affiliations for an 'aesthetic maximum' of six authors.
% Add names, affiliations, addresses for
% the seventh etc. author(s) as the argument for the
% \additionalauthors command.
% These 'additional authors' will be output/set for you
% without further effort on your part as the last section in
% the body of your article BEFORE References or any Appendices.

\numberofauthors{2} 

\author{
% You can go ahead and credit any number of authors here,
% e.g. one 'row of three' or two rows (consisting of one row of three
% and a second row of one, two or three).
%
% The command \alignauthor (no curly braces needed) should
% precede each author name, affiliation/snail-mail address and
% e-mail address. Additionally, tag each line of
% affiliation/address with \affaddr, and tag the
% e-mail address with \email.
%
% 1st. author
\alignauthor
Andrew Meyer\\
       \affaddr{Bowling Green State University}\\
       \affaddr{Bowling Green, Ohio, USA}\\
%       \affaddr{Address continued}\\
       \email{apmeyer@bgsu.edu}
% 2nd. author
\alignauthor
Sankardas Roy\\
       \affaddr{Bowling Green State University}\\
       \affaddr{Bowling Green, Ohio, USA}\\
%       \affaddr{Address continued}\\
       \email{sanroy@bgsu.edu}
% 3rd. author
%\alignauthor 
%Third Author\titlenote{You can use a ``titlenote'' to recognize your advisor.}\\
%       \affaddr{Affiliation}\\
%       \affaddr{Address}\\
%       \affaddr{Address continued}\\
%       \email{third.author@email.invalid}
%
%\and  % use '\and' if you need 'another row' of author names
%
}

% There's nothing stopping you putting the seventh, eighth, etc.
% author on the opening page (as the 'third row') but we ask,
% for aesthetic reasons that you place these 'additional authors'
% in the \additional authors block, viz.
%\additionalauthors{Additional authors: John Smith (The Th{\o}rv{\"a}ld Group,
%email: {\texttt{jsmith@affiliation.org}}) and Julius P.~Kumquat
%(The Kumquat Consortium, email: {\texttt{jpkumquat@consortium.net}}).}
%\date{30 July 1999}
% Just remember to make sure that the TOTAL number of authors
% is the number that will appear on the first page PLUS the
% number that will appear in the \additionalauthors section.

\maketitle
\begin{abstract}
Digital forensics (DF) tools are used for post-mortem investigation of cyber-crimes and cyber-attacks. 
National Institute of Standards and Technology (NIST) 
has set standards for DF tools. Compliance of the standards by the DF tools is critical, especially in judical proceedings. 
In this paper we consider standardization of one class of DF tools that are for Deleted File Recovery (DFR). 
Our experiments with a popular tool-suite (named Autopsy) show that it does not satisfy some of the standards for DFR. 
Furthermore, we evaluate other frequently-used DFR tools on the same standards, and report the findings.  
We compile a comparative analysis of these tools' performance, which could help the user choose the right DFR tool. 
%We hope our findings will trigger more research activities on compliance of standards from the researcher community as well as the practitioners. 
\end{abstract}


\section{Introduction}

Both in corporate and government settings, digital forensic (DF) tools are used for post-mortem investigation of cyber-crimes and cyber-attacks. 
Nowadays it is common \cite{df:news} for the police to use DF tools to follow an eletronic trail of evidence to track down the suspect. 
To maintain the quality and integrity of DF tools, National Institute of Standards and Technology (NIST)'s 
Computer Forensics Tool Testing Program (CFTT) \cite{cftt:nist} 
set standards for these tools. Maintaining the standards of DF tools 
is especially critical for judicial proceedings: Usage of a forensic tool that does not follow the standards may cause evidence to be thrown 
out in a court case whereas incorrect results from a forensic tool can also lead improper prosecution of an innocent defendant. 

The focus of this paper is about standardization of one class of DF 
tools that are for Deleted File Recovery (DFR) \cite{meta:dfr:standards}. 
As the name suggests, a DFR tool attempts to retrieve deleted files
from a file system of a computer. As an example, given a hard disk or a USB drive 
(which might have been seized from the suspect computer or collected from the crime scene), a 
forensics professional can use a DFR tool to investigate about (and potentially retrieve) deleted files that 
a suspect may have deleted to hide important information. 
The success or failure of a DFR tool can decide the outcome of a case.  

Our experiments with a popular DF tool suite (named Autopsy \cite{autopsy}) 
show that it does not satisfy all NIST standards for DFR. 
Furthermore, we extensively experimented with other frequently used DFR tools. 
We compare those tools' performance and compile a comparative analysis, which could help the user choose the right DFR tool. 

Evaluating the performance of a DFR tool is complex because many elements of a forensics scenario determine 
the success or failure of the file recovery process. 
A few such factors are the type of the file system (FAT, NTFS, etc.), presence of other active/deleted 
files in the file system, fragmentation of a file, a deleted file being overwritten by another file, and so on.
So, comparison of two DFR tools is scientific only if they are compared while keeping these factors same. 
Via extensive analysis, we design a set of test file system images (for either of FAT and NTFS) which considers each of the above factors independently. 
We claim that this list of test cases is exhaustive and thus claim that our evaluation gives a complete picture. 
 

The main contributions of the paper are listed as follows:
\begin{itemize}
\item We design and build an exhaustive list of canonical test file system (FAT and NTFS) images to test the DFR tool per NIST standards. 
\item We perform evaluation of frequently-used DFR tools (including free tools as well as proprietary ones) on the test images.
\item For the interesting cases of tools' success or failure, we provide logical explanation.
\item We provide critique on applicability of some of the NIST standards in a practical setting. 
\end{itemize}


The NIST CFTT portal currently publishes reports of only a subset of DFR tools. 
However, the scope of tools needs to be expanded as many new tools come to market and become popular.
Also, existing DFR tools should be retested to ensure their reliability is consistent 
as new patches and features come out. 
Adding new reports to the CFTT website will allow tool developers a 
chance to continually develop their tool for the better. We will submit our study reports to the CFTT portal.

As a side benefit, our work leads to a few hands-on lab-modules to be used in digital forensics courses 
at BGSU, enriching the new DF specialization program in the CS department. We will also make these modules
available for relevant instructors at other universities.


\section{Research Questions}

A DFR tool is a piece of software that can retrieve (residual data of) a file that was deleted 
from a storage device (e.g., computer hard disk, flash drive, or so). We evaluate a set of 
popular DFR tools on the scale of CFTT standards. 
In particular, below are the research questions (RQs) that we target to answer. 

\begin{itemize}
\item[RQ1.] Do the popular DFR tools (as available in the market) meet the NIST CFTT standards? 
If not, which tool meets which part of the standard? 

\item[RQ2.] What factors make the tools fail or succeed?

\item[RQ3.] Are the free DFR tools more effective compared to the enterprise-level (proprietary) tools?
\end{itemize}

The identification of errors, such as for not recovering a deleted file or attempting to recover a file that was never there 
(Type I and Type II errors, respectively), is an important metric for a DFR tool. 
Type I and Type II errors account for majority of the standard. Many factors impact the performance of a DFR tool, 
including the condition of the file system (e.g., the file system being almost full or empty) and 
how the file was deleted (being placed in Recycle bin, permanent deletion, reformat of disk, or so). 
We consider these variables in the design of experiment when we compare the tools. 
In addition to core features of CFTT standards \cite{cftt:nist}, we also use optional features \cite{cftt:nist} for comparing the DFR tools.



\section{Background}

\subsection{Metadata-Based Deleted File Recovery}

\subsection{FAT Filesystem}

\subsection{NTFS Filesystem}

\subsection{NIST Standards}
\begin{enumerate} % TODO cite standards document
 \item ``The tool shall identify all deleted File System-Object entries accessible in residual metadata.''
 \item ``The tool shall construct a Recovered Object for each deleted File System-Object entry accessible in residual metadata.''
 We consider a tool passing this standard as long as it outputs a file for each deleted file, even if the output file is empty.
 \item ``Each Recovered Object shall include all non-allocated data blocks identified in a residual metadata entry.'' For FAT filesystems, we consider a tool passing this standard if it recovers at least the first contiguous segment of unallocated sectors starting from the first sector originally allocated to the deleted file. For NTFS filesystem, the tool must recover all unallocated sectors originally allocated to the deleted file.
 \item ``Each Recovered Object shall consist only of data blocks from the Deleted Block Pool.''
 We consider a tool passing this standard if all recovered sectors were all originally allocated to the deleted file, and had not been reallocated to any other file.
\end{enumerate}


\section{Approach}

\subsection{Overview}

To test the DFR tools, we first designed hypothetical test scenarios to simulate the challenges of real-world file recovery.
We then created each scenario in real file systems and saved them as raw images. 
Using the images as input, we ran each DFR tool and attempted to recover all deleted files. 
Finally, we compared the recovered files to their original versions in order to judge the tools' 
compliance with the NIST standards). A high-level view of the methodology for a typical test case is illustrated in Figure \ref{fig:overview}.

\begin{figure}[h]
    \centering
    \includegraphics[width=\linewidth]{fig/overview.png}
    \caption{A filesystem containing deleted files is created on the external drive. That filesystem is then given as input to a DFR tool, which attempts to recover the deleted files. The recovered files are then analyzed to judge the DFR tool's compliance with the NIST CFTT standard.}
    \label{fig:overview}
\end{figure}

\subsection{Designing Recovery Scenarios}
To test the DFR tools' compliance with the standards, we designed a variety of scenarios in which a tool might have to recover a deleted file. 
We started with the simplest possible case: a file system containing just one deleted file. 
This case is ideal and trivial, but by adding more files, we can create a greater variety of scenarios.

The NIST standards limit their scope to recovery of files which were ``created and deleted in a process similar to how an end-user would create and delete files,''\cite{meta:dfr:standards} and exclude ``files and file system metadata that is specifically corrupted, modified, or otherwise manipulated to appear deleted.''\cite{meta:dfr:standards}
In other words, these standards address situations in which files were deleted via normal file system operations as implemented by a typical operating system, rather than manual tampering with the files and file system.
Within these constraints, there are two factors which can significantly complicate the file recovery process: 
fragmentation, and overwriting. 

These factors are thus the foundation of our test scenarios, with all cases besides the first involving either fragmented files, overwritten files, or a combination of both. 
The goal is to create test cases which are canonical; in other words, they constitute the basic elements of a file recovery scenario.
We suggest such a canonical list of test cases should be considered representative of all possible scenarios within the scope of the standards.

We designed the following test cases:
\begin{itemize}
    \item [1] Single deleted file
    \item [2] Deleted file fragmented around an active file (Illustrated in Figure \ref{fig:case_2})
    \begin{figure}[h]
        \centering
        \includegraphics[width=\linewidth]{fig/case2.png}
        \caption{Test Case 2}
        \label{fig:case_2}
    \end{figure}
    \item [3] Deleted file fragmented around a deleted file
    \item [4i] Beginning of deleted file overwritten by an active file
    \item [4ii] Middle of deleted file overwritten by an active file (Illustrated in Figure \ref{fig:case_4ii})
        \begin{figure}[h]
        \centering
        \includegraphics[width=\linewidth]{fig/case4ii.png}
        \caption{Test Case 4ii}
        \label{fig:case_4ii}
    \end{figure}
    \item [4iii] Deleted file partially overwritten by an active file which doesn't end on a sector boundary
    \item [4iv] Deleted file entirely overwritten by an active file
    \item [5i] Beginning of deleted file overwritten by a deleted file
    \item [5ii] Middle of deleted file overwritten by a deleted file
    \item [5iii] Deleted file partially overwritten by a deleted file which doesn't end on a sector boundary
    \item [5iv] Deleted file entirely overwritten by a deleted file
    \item [6] Deleted file fragmented around an active file, with the second fragment overwritten by an active file (Illustrated in Figure \ref{fig:case_6})
    \begin{figure}[h]
        \centering
        \includegraphics[width=\linewidth]{fig/case6.png}
        \caption{Test Case 6}
        \label{fig:case_6}
    \end{figure}
    \item [7] Deleted file fragmented around an active file, with the second fragment overwritten by a deleted file
    \item [8] Deleted file fragmented from the end of the file system to the beginning
    \item [9] Deleted file fragmented from the end of the file system to after an active file
    \begin{figure}[h]
        \centering
        \includegraphics[width=\linewidth]{fig/case10.png}
        \caption{Test Case 10}
        \label{fig:case_10}
    \end{figure}
    \item [10] Deleted file fragmented from the end of the file system to after a deleted file (Illustrated in Figure \ref{fig:case_10})
\end{itemize}

Because NTFS keeps track of the locations of all parts of a file even after deletion, fragmentation is not particularly interesting. Cases 8, 9, and 10 would be redundant with case 2, so we have excluded them for NTFS. Due to how NTFS allocates space for files, cases 4ii and 5ii cannot occur as a result of normal file operations, so they have also been excluded. No cases are excluded for FAT tests.

\subsection{Creating Test Images}
% Step by step process
All test file systems were created in partitions on a 32 GB flash drive. For each test case, the first step is to entirely write over the partition with zeros. This ensures all cases start from identical, reproduceable conditions. A new file system is written to the partition, then new files are written to the file system and deleted. The files used are simple text files containing one letter repeated (e.g. ``aa1M'' is 1 MiB of the letter 'a'). Files are written to the test file system by simply copying them from another drive. In some cases we also append data to a file in the test file system. Once the test file system matches the intended scenario, a read-only image of the partition is created. All tests are performed on these images rather than the original drive.

\subsubsection{Challenges}
% Caching problem
It is important to consider when creating test images that the low-level behavior of file operations is not always obvious. For example, when writing a file, there is no guarantee the file's data will be immediately written to the disk. The operating system may cache the operation and wait until the optimal time to perform the write, in order to maximize system performance. We observed this early on, as writing a file and subsequently deleting it would always result in the file's metadata being written, but often left no evidence of the file's data having ever existed. This behavior is obviously undesireable because it leaves nothing meaningful to be recovered. We resolved this by calling the ``sync'' system call, which causes any such cached data to be immediately written to the disk, in between file writes and deletions. Unmounting the file system has a similar effect.

% Learning and using the allocation algorithms
Another type of low-level behavior relevant to the image creation process is the allocation algorithm. The operating system must have some kind of algorithm to decide where in the data area new files should be written. Common allocation algorithms include ``first available,'' ``next available,'' and ``best fit.''
Learning and understanding whatever algorithm the OS uses is very helpful for forcing a specific arrangement of files. We observed that when writing to a FAT file system, Linux uses a ``next available'' algorithm. After the file system is mounted, the first write will start at the first free space in the data area. The next file will be written starting from the first free space after the previous file.
Meanwhile, when writing to an NTFS file system, Windows 10 appears to use a ``best fit'' algorithm. In this case, Windows tries to find the smallest space in which the file can fit without being fragmented, and write it there.

% Directory entries problem

\subsection{Recovering Files}

The settings we used when testing each tool are as follows:
For Autopsy, we performed a standard recovery with all ingest modules disabled.
For Recuva we perfomed a standard recovery using the free version with default settings.
For FTK Imager, we performed a standard recovery using the free version with default settings.
For TestDisk we used the ``file undelete'' feature under ``Advanced Filesystem Utils.''
For Magnet Axiom we performed a ``full scan'' in Axiom Process and exported all files accessible in ``Filesystem View'' in Axiom Examine.

\subsection{Results}

After testing each tool, we analyzed the recovered object(s) from each test case. 
If the recovered file is identical to the original, obviously all standards have been met. 
While this is ideal, it is often impossible to perfectly recover a file (such as when it is overwritten) so the standards do not require it. 
In our results, the file is only ever recovered perfectly in FAT cases 1 and 2, and NTFS cases 1-3. 
For all other cases, the tool is judged on each core feature individually. 
These judgements are summarized in Figure \ref{fig:results_fat} for FAT test cases and Figure \ref{fig:results_ntfs} for NTFS test cases.

\begin{figure*}[h!]
    \centering

    \begin{subfigure}{0.17\linewidth}
        \includegraphics[width=\linewidth]{fig/autopsy_results_fat.png}
        \subcaption{Autopsy}
    \end{subfigure}~~
    \begin{subfigure}{0.17\linewidth}
        \includegraphics[width=\linewidth]{fig/recuva_results_fat.png}
        \subcaption{Recuva}
    \end{subfigure}~~
    \begin{subfigure}{0.17\linewidth}
        \includegraphics[width=\linewidth]{fig/ftk_results_fat.png}
        \subcaption{FTK}
    \end{subfigure}~~
    \begin{subfigure}{0.17\linewidth}
        \includegraphics[width=\linewidth]{fig/testdisk_results_fat.png}
        \subcaption{TestDisk}
    \end{subfigure}~~
    \begin{subfigure}{0.17\linewidth}
        \includegraphics[width=\linewidth]{fig/axiom_results_fat.png}
        \subcaption{Magnet Axiom}
    \end{subfigure}~~
        
    \caption{Test results on FAT test cases for each DFR tool. Rows represent test cases whereas columns represent NIST core features. Blue is passing, red is failing, grey is not tested.}
    \label{fig:results_fat}
\end{figure*}

\begin{figure*}[h]
    \centering

    \begin{subfigure}{0.17\linewidth}
        \includegraphics[width=\linewidth]{fig/autopsy_results_ntfs.png}
        \subcaption{Autopsy}
    \end{subfigure}~~
    \begin{subfigure}{0.17\linewidth}
        \includegraphics[width=\linewidth]{fig/recuva_results_ntfs.png}
        \subcaption{Recuva}
    \end{subfigure}~~
    \begin{subfigure}{0.17\linewidth}
        \includegraphics[width=\linewidth]{fig/ftk_results_ntfs.png}
        \subcaption{FTK}
    \end{subfigure}~~
    \begin{subfigure}{0.17\linewidth}
        \includegraphics[width=\linewidth]{fig/testdisk_results_ntfs.png}
        \subcaption{TestDisk}
    \end{subfigure}~~
    \begin{subfigure}{0.17\linewidth}
        \includegraphics[width=\linewidth]{fig/axiom_results_ntfs.png}
        \subcaption{Magnet Axiom}
    \end{subfigure}~~
        
    \caption{Test results on NTFS test cases for each DFR tool. Rows represent test cases whereas columns represent NIST core features. Blue is passing, red is failing, grey is not tested.}
    \label{fig:results_ntfs}
\end{figure*}

For cases in which a tool does not fulfull core feature 1, in other words, it cannot find a deleted file, we make no judgement about the remaining core features.

\subsubsection{Recovering Fragmented Files}
In cases of fragmentation in FAT file systems, we found each tool generally approaches recovery in one of two ways. 
Recuva and Magnet Axiom start from the beginning of the file and recover the full length of the file even if an active file exists in that space. 
Autopsy, FTK, and TestDisk will start from the beginning of the file and recover the full length, but skip over any active files they encounter.
Autopsy, FTK, and TestDisk recover all of file A, while Recuva and Magnet Axiom's recovered images erroneously contain data from file B, causing them to fail core feature 4. 
When the space in between fragments is unallocated, all tools recover the file as though it was contiguous, pulling some erroneous data and failing core feature 4. 
When the fragmentation occurs at the end of the file system, Recuva, FTK, and TestDisk recover only the first fragment, while Autopsy returns a short file of null data, and Magnet Axiom reports an error and returns an empty file.
Cases with fragmentation are trivial for NTFS file systems as more information is available from the metadata. 
Unsurprisingly, no tools had problems with fragmentation cases for NTFS.

\subsubsection{Recovering Overwritten Files}
In cases where a file has been overwritten by an active file, we found most tools recover the deleted file as though it is not overwritten, failing core feature 4. 
A few exceptions are FTK Imager, which recovers the file up to the point where it has been overwritten, and Autopsy, which generally recovers only the first cluster of an overwritten file in FAT, and behaves like the other tools for NTFS. 
TestDisk also exhibits the same behavior as FTK for FAT case 4ii only. 
Strangely, Magnet Axiom's recovered objects for FAT cases 4i and 4ii include the overwritten sections, but nothing after them.
Other Magnet Axiom results were similar to the other tools.
When the overwriting file has also been deleted, all tools recover the first file as though it is not overwritten.

\subsubsection{Abnormal Results}
A few results stand out as unusual.
These are cases for which it is difficult to infer from the recovered object what approach a tool is using.

For FAT cases 4ii, 6, 8, 9, and 10, Autopsy returns a 1.5 KiB file of null data.
1.5 KiB is equivalent to 3 sectors, while a FAT cluster in our cases is equivalent to 4 sectors or 2 KiB.

TestDisk fails to identify a file for NTFS cases 4iii and 4iv only. 
These are the only test cases in which a tool does not fulfill core feature 1.

For FAT cases 4i and 4ii, Magnet Axiom does not recover the entire length of the deleted file, but it also does not exclude the overwritten sections. 
In both cases, it recovers up to the end of the overwritten sections, rather than up to the beginning like FTK does.



\section{Discussion}

\subsection{Answering Research Questions}

\begin{paraphrase}
 \subsubsection{RQ1}
Do the popular DFR tools meet the NIST CFTT expectation? 
If not, which tool meets which part of the expectation? 

Generally, we found that the DFR tools we tested did not consistently meet the NIST CFTT expectation.
TestDisk failed to fulfill DFR-CR-01 because it did not identify deleted files in two test cases.
All tools fulfilled DFR-CR-02, as they produced a recovered object for every deleted file they identified.
Autopsy and Magnet AXIOM failed to fulfill DFR-CR-03 because in several test cases they did not recover data they had access to.
All tools failed to fulfill DFR-CR-04 because in many cases they recovered data which was not part of the original file.

\subsubsection{RQ2}
What factors make the tools fail or succeed?

The most common factor causing tools to fail is when a deleted file has been overwritten.
Core feature 4 requires that a tool only recover data which was originally a part of the deleted file.
A tool's success regarding this feature is thus a measure of its restraint.
The only tool to consistently fulfill DFR-CR-04 is FTK Imager.
When it detects that a file has been partially or completely overwritten by another file, it omits the deleted sections (and everything after them in FAT).
However, in cases when the overwriting file has also been deleted, even FTK fails to fulfill this core feature.
It is worth noting that Autopsy does appear to react to overwritten files; for some cases of overwriting in FAT, it returns only a single cluster, presumably the starting cluster of the deleted file.
However, since that cluster has been overwritten, Autopsy still fails to fulfill DFR-CR-04 in those cases.

Another factor that causes multiple failures is simulated in FAT cases 8, 9, and 10.
In FAT, a file can be written starting close to the end of the file system, without enough space to fit contiguously.
In such cases, the file must be fragmented, and since it is already at the end of the file system, the second fragment will appear closer to the beginning of the file system (this is illustrated in Figure \ref{fig:case14}).
This scenario could realistically occur when the file system is almost full.
In these cases, no tool is able to recover the second fragment of the deleted file; however, because FAT fragmentation is unpredictable, we only require them to recover the first fragment.
Interestingly, Autopsy and Magnet AXIOM fail to recover anything at all, with Autopsy returning a short file of null data, and Magnet AXIOM returning an empty file after displaying an error message.
\end{paraphrase}

\subsubsection{RQ3}

\TODO{TODO}
% PhotoRec gets only full pass

\subsubsection{RQ4}

\TODO{TODO}
% Foremost and Magnet AXIOM don't find any TIFs, TODO does Magnet AXIOM support TIF carving?
% Tools that carve a lot of files (scalpel) do bad on CR5, reference discussion
% PhotoRec is really bad with out-of-order carving
% PhotoRec has excellent error checking, almost only carves valid files
% PhotoRec has a harder time with empty speace than adjacent files, only tool to have different results between basic and nofill
% Simple cases are easy, fragmentation and overwriting are harder
% Disorder seems to be the hardest case, this makes sense

% \subsection{Hybrid Tools} probably just cover this in the approach section

\subsection{Ambiguity in NIST Guidelines}
% Metadata
\subsubsection{FAT Fragmentation and Metadata-Based Tools}
\begin{paraphrase}
Core feature 3's requirement that a tool recover ``all non-allocated data blocks identified in a residual metadata entry''\cite{meta:dfr:standards} is not well-defined when considering a FAT file system. 
In FAT, the only relevant metadata left over after file deletion is the address of the first cluster of file data, and the file's length. 
If the deleted file is fragmented at any point, no evidence remains in the metadata. 
Therefore, interpreting the wording very closely, a tool is only required to recover the first cluster of a file's data. 
As this would not be particularly useful, it is unlikely that this was the intended meaning. 
For these tests we interpret DFR-CR-03 as requiring the first contiguous segment of unallocated clusters starting from the first cluster originally allocated to the deleted file. 
In other words, if the file is fragmented, the tool must recover at least the first fragment. 
If a file is partially overwritten, the tool must recover at least the clusters before the overwritten part.
In essence, the tool is only required to recover data for which it does not have to guess what file the data belongs to.
However, it should be emphasized this is an assumption and the intent of the standard in this case needs clarification.
\end{paraphrase}

\subsubsection{Contradictory Core Features for Metadata-Based Tools}
\begin{paraphrase}
When designing test cases, we found situations in which core features 3 and 4 are entirely incompatible. 
Core feature 3 specifies ``all non-allocated data blocks identified in a residual metadata entry,''\cite{meta:dfr:standards} but that can sometimes still include data from other files. 
One such situation is when a deleted file is overwritten, and then the overwriting file is also deleted, such as in case 5i (as seen in Figure \ref{fig:case8}).

Assuming the file system is NTFS (to avoid the aforementioned ambiguity with DFR-CR-03 and FAT), the residual metadata entry for File A (in this case its Master File Table entry) should list every cluster File A once occupied. 
All of those clusters are unallocated, so to fulfill DFR-CR-03, the tool must recover them. 
However, some of those clusters have been overwritten by File B. Core feature 4 requires that a tool only recover ``data blocks from the Deleted Block Pool,''~\cite{meta:dfr:standards} and defines the Deleted Block Pool as all blocks which ``have not been reallocated or reused.''~\cite{meta:dfr:standards}
Core feature 3 would require tools to recover the clusters reused by File B, while DFR-CR-04 would forbid this. 
It could be argued that the tool should use File B's metadata to recognize that File B overwrote File A, but this is not always realistic. 
While the file system stores information such as creation and modification times, this is not ``essential metadata,'' meaning it is not involved in the operation of the file system, so operating systems may implement it inconsistently, or not at all.~\cite{carrier:filesystems}
Since the time information cannot be counted on to be reliable, there is no way to know for sure which file overwrote which. 
It is also possible for File B's metadata entry to be overwritten at some point before File A's, in which event there is no way for the tool to know File B even existed.

The standards document acknowledges that the ``potential for corruption [is] inherent with data that is no longer maintained by a file system,''~\cite{meta:dfr:standards} and that the recovered object ``may not completely match the original FS-Object.''~\cite{meta:dfr:standards}
We propose the standards themselves should be revised to better account for such situations.
This could mean adding an exception to either the third or fourth core feature, for cases in which data blocks are overwritten and subsequently deallocated.

NIST's James Lyle proposes that rather than deal with the complications involved in accommodating 
many different file systems, standards should be written for ``an ideal file system that leaves in 
residual metadata  all  information  required  to  reconstruct  a  deleted  file''~\cite{lyle2011-ICDF2C}. 
While this can result in tools being held to an impossible standard when using certain file systems, 
Lyle says that is acceptable because the user experience will be the same regardless of whether a feature is impossible or has merely been left out.
If the NIST guidelines were created with such an ideal file system in mind, the current standards may be adequate.
However, that philosophy should be clarified in the guidelines document to avoid confusion.
\end{paraphrase}



% Carving
\subsubsection{False-Positives from File Carving}
% What to do about additional files? It's clear for CR1, but not for the others.

Core features 2-5 for file carving tools establish requirements for each ``carved file.''
The guidelines document defines a carved file as ``A file created by a carving tool purported to be one of the source files present in the search arena.''~\cite{carving_standards}
This means even false positives which are not part of one of the original files affect a tools' evaluation on four out of five of the core features.
We suggest that this results in misleading and less informative results, especially when those results are used to compare different tools.
 
Interpreting ``each carved file'' to include even the ones with no relation to the original files results in dramatically low scores for tools like Scalpel (which carved at least 50 additional ``files'' in most of our tests), while favoring tools which are more conservative in their recovery.
However, this would would obscure a relevant trade-off, that a more aggressive tool will have a high false-positive rate, but may recover files that a more conservative tool would miss.
An investigator may have this trade-off in mind when selecting a DFR tool, so the NIST guidelines should not make a tool that sits on one side of that trade-off look objectively worse than others.
As the standards are currently written, a tool that carves only one file from an image, but recovers it correctly, would be considered perfect on four out of five core features.
Meanwhile, a tool that perfectly carves all 40 files from an image, but also returns 150 false-positives, would likely score very poorly for CF-CR-03, CF-CR-04, and CF-CR-05.
It would score especially poorly on CF-CR-05 as false-positives will almost never be a valid file.
Since the NIST guidelines do not directly account for the false-positive rate, it indirectly and disproportionately affects several core features, diluting their usefulness.

To resolve this issue, we propose the following changes to the NIST guidelines:
\begin{arabiclist}
 \item Add a new definition: \emph{Positive Carved File}, a carved file which corresponds to a supported file header signature from a source file that is present in the search arena. \TODO{come up with a better name for this}
 \item In CF-CR-03, CF-CR-04, and CF-CR-05, change ``carved file'' to ``positive carved file.''
 \item Add an additional core feature: \emph{The tool shall not return any carved files that do not correspond to a supported file header signature from a source file that is present in the search arena.} This could be scored as the ratio of positive carved files to total carved files.
\end{arabiclist}

The intent of these changes is to somewhat atomize the guidelines, so each core feature evaluates a tool on a single capability.
This should make the trade-offs of certain tools more apparent, enabling investigators to make more informed and nuanced tool choices based on the capabilities that are most important for their use case.

\subsection{Related Work}

\begin{paraphrase}
 Arthur et al. published an article~\cite{arthur2004} in 2004 which analyzes several DF (digital forensics) tools, including FTK Imager.
While the tools are judged based in part on file recovery capabilities, the article does not present how these judgments were reached.
The article also addresses DF tools' disk imaging and hashing functionalities.

James Lyle from NIST published an article~\cite{lyle2011-ICDF2C} in 2011, which lays out a strategy to evaluate the metadata-based DFR tools. To the best of our knowledge, 
his is one of the first works that identified some of the challenges in setting standards for evaluation of metadata-based DFR tools.
In our understanding, NIST considered the above findings~\cite{lyle2011-ICDF2C} while they set the guidelines for metadata-based DFR tools.  
The NIST guidelines~\cite{meta:dfr:standards} are publicly available on the NIST CFTT portal~\cite{cftt:nist}, which we have used in the current work.

Recently, Loja et al.~\cite{loja2016} analyzed a variety of DF tools, including Autopsy and FTK. 
They discussed a wide range of DF tools, not just DFR tools, and compared them on metrics such as price and supported features. 
In contrast with our paper, their work does not follow a specific standards document and takes a more general approach instead.
Furthermore, B. V. Prasanthi~\cite{prasanthi2016} presented a general review of DF tools. 
In particular, Prasanthi summarizes the features of several tools but does not make any claims about standards compliance of specific tools (which is contrast with our work).
It~\cite{prasanthi2016} also includes a variety of DF tools besides DFR tools.
\TODO{Do these already cover carving? Are there any papers specifically on carving we should cite?}
\end{paraphrase}


\subsection{Future Work}

In addition to the four core features, there are several optional features listed in the NIST CFTT guidelines for metadata-based DFR tools~\cite{meta:dfr:standards}.
Future work could extend our methodology to evaluate metadata-based tools on these optional features.
Our evaluation of metadata-based tools was limited in scope to just the FAT and NTFS file systems, making it somewhat Windows-centric.
As investigators need to be able to recover evidence from Linux or MacOS devices, futrure work could evaluate metadata-based tools on ext4, HFS, and other common file systems.
It is common for files to be embedded within other files, for example thumbnails in some graphical formats.
Future work could test the ability of file carving tools to recover these files, and interpret and critique the NIST guidelines in the context of embedded files.
Future work could also extend our investigation of file carving tools beyond just graphical file formats, such as to video or document file formats, which may be equally valuable to an investigation.


\section{Conclusion}

%Papers should be no more than ten pages, single- spaced, with 12pt font
%for the text content. Left, right, top and bottom margins need to be 1''.  Pages
%should not be numbered (they will be numbered for the proceedings). The title
%and author information are single-column centered at the top of the first page.   
%
%The remaining document should be double- column format. Please give an address
%and email address for each author of the paper. Sections (if used) should be
%numbered and titled. Figures, tables, and other diagrams should be numbered and
%titled. Please try to include each figure near the text it addresses. If
%necessary, you may elect to place large figures at the end of your paper using
%the full width of the page (single-column format). 
%
%Follow the rules of accepted grammar and mathematical formatting. Remember that
%the quality of your paper is one of the criteria for judgment. 

%\subsection{References}
%Use the standard CACM format for references -- that is, a numbered list at the
%end of the article, ordered alphabetically by first author, and referenced by
%numbers in brackets~\cite{bowman:reasoning}.  The references are also in 12pt,
%but that section is ragged right.  Other references that may prove helpful
%include~\cite{Dupre:1995:BW,Lamport:LaTeX,Mittelbach:2004:LC}.
%
%\subsection{Page Number, Headers and Footers}
%Do not include headers, footers, or page numbers in your submission.  These will
%be added when the proceedings are assembled.

%%% To balance the columns, we need this to be in the code for the LEFT column
% on the last page
\balance

%ACKNOWLEDGMENTS are optional
\section{Acknowledgments} A. Meyer's work has been partially supported by a grant from BGSU's Center for Undergraduate Research and Scholarship 
(CURS) in summer of 2019. S. Roy's work has been partially supported by a NIST grant (2017-19) that he has been awarded with as a Co-PI. 
% This section is optional; it is a location for you to acknowledge grants,
% funding, editing assistance and what have you.  In the present case, for
% example, the authors would like to thank the maintainers of ACM's conference
% proceeding \LaTeX\ template from which this document is directly derived and the
% past organizers of the MCURCSM conference.

%
% The following two commands are all you need in the
% initial runs of your .tex file to
% produce the bibliography for the citations in your paper.
\bibliographystyle{plain}
\bibliography{mybib}  % mybib.bib is the name of the Bibliography in this case
% You must have a proper ".bib" file
%  and remember to run:
% latex bibtex latex latex
% to resolve all references
%


% That's all folks!
\end{document}
