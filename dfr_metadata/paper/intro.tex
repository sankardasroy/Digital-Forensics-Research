
\section{Introduction}

Both in corporate and government settings, digital forensic (DF) tools are used for post-mortem investigation of cyber-crimes and cyber-attacks. National Institute of Standards and Technology (NIST)’s Computer Forensics Tool Testing Program (CFTT) [1] proposed standards for DF tools [2] to help determine their quality and integrity. Maintaining the standards of DF tools is especially critical for judicial proceedings: Usage of a forensic tool that does not follow the standards may cause evidence to be thrown out in a court case whereas incorrect results from a forensic tool can also lead improper prosecution of an innocent defendant. The focus of our proposed work is about standardization of one class of DF tools that are for Deleted File Recovery (DFR). Our preliminary experiments at BGSU with a popular DF tool suite (named Autopsy [3]) show that it does not satisfy multiple standards for DFR. In the proposed project, we plan to expand (and complete) our experiments with Autopsy and other frequently used DFR tools, and we aim to bring our findings to the notice of the researcher community as well as the practitioners. We will also compare those tools’ performance and publish the results, which will help the user choose the right DFR tool. Our work will also result a few hands-on lab-modules for the future students at BGSU, enriching the new DF specialization program in the CS department.

