\vspace{-.1in}
\section{Related Work}
\label{sec:related}
\vspace{-.02in}

The \genome~\cite{zhou2012dissecting} %  is the closest related work of the current paper. 
% This
was the first research project that has provided the community 
an Android malware dataset. 
%In particular, it contains 1260 malware apps categorized in 49 families, which were discovered in 2010 and 2011. 
This dataset has been the only well-labeled one and has been widely studied 
and used by the research community.
Unfortunately, it has not been updated after its creation time around 2011.
We comparatively studied this dataset with the new malware samples we have, 
and found that the Genome dataset does not include many of the new threats, 
which motivated us to carry out this work. Our dataset also provides much more
detailed information on Android malware behaviors than that in Genome. Moreover,
we provide detailed documentation of the process used in creating the dataset, 
including the guidelines for the manual analysis, to help other researchers
do the same.
%In short, Genome dataset no longer represents the current Android malware world, which has motivated us to do the current work. 

Recently, the \emph{AndroZoo}~\cite{allix2015androzoo} dataset has been published, 
which contains more than 3 million Android apps 
from Google Play, other smaller markets, and app repositories.
\emph{AndroZoo}'s goal is to create a comprehensive app collection for software 
engineering studies. Our goal is different and we focus on (only) malware apps 
to study their security related behaviors. Our dataset provides malware labels
and detailed behavior information of the malware. 
% As expected, the majority of the apps in \emph{AndroZoo} are benign apps. 
% \emph{AndroZoo} did not yet report the behaviors of the malware apps in their dataset. 

There are a few other repositories for Android malware apps 
which researchers can use, such as \emph{Contagio Minidump}~\cite{minidump} 
and \emph{VirusShare}~\cite{virusshare}. However, they do not provide 
a comprehensive malware collection or comprehensive label and behavior information on
the malware.

The \emph{ANDRUBIS}~\cite{lindorfer2014andrubis} combines static and dynamic analysis
to automatically extract feature and behaviors from Android apps, and
studies the changes in the malware threat landscape and trends among ``goodware,'' or
benign apps, developers.
However, as many behaviors are either unknown or can evade the automated analysis method,
this work cannot give a comprehensive understanding of the malware landscape as we 
produced through the systematic and deep manual analysis.

\emph{AVclass}~\cite{sebastian2016avclass} provides a method to extract malware family name
by processing the AV labels obtained from VirusTotal. 
We adopted a similar approach for identifying malware family label.
Our work is focused on deep manual analysis of malware samples from different malware varieties, 
and reporting the detailed behavioral profiles for Android malware.

% independently developed without
% the knowledge of the AVclass work. The difference between own approach is discussed in
% Section~\ref{sec:data:family}. Furthermore, we provide the variety information for each malware family.

% Few recent works include \emph{Amandroid} \cite{wei2014amandroid}, \emph{DroidSafe} 
% \cite{gordon2015information}, \emph{FlowDroid} \cite{arzt2014flowdroid}, and \emph{IccTA} \cite{li2015iccta}.
% Future detection tools can utilize our malware dataset to test and improve the effectiveness. 

There has been quite some work on how to detect malicious apps. 
The \emph{Drebin}~\cite{arp2014drebin} work applies machine learning (ML) techniques to Android malware
detection.
The authors made the set of \emph{feature vectors} used in the ML work available to the community.
More recently, \emph{MassVet}~\cite{chen2015finding} provides a method
to detect malware apps by observing the repackaging traits (if any) 
compared to that of other apps. % However, \emph{MassVet} does not target other types (\ie standalone) 
% of malware.
Rastogi, \etal~\cite{rastogi2016these} conducted research on identifying adware tricks and drive-by-download techniques.
\emph{Harvestor}~\cite{rasthofer2016harvesting} attempts to extract the 
\emph{run-time} values from obfuscated apps
to detect malware. Researchers have identified ways in which Android users 
can be deceived to misidentify a malicious app window as a legitimate app's~\cite{bianchi2015app}.
Moreover, \emph{CopperDroid}~\cite{tam2015copperdroid} is a dynamic analysis system which attempts 
to reconstruct the behaviors of Android malware.
Our work complements these and other Android malware analysis work by providing a comprehensive 
dataset of Android malware with detailed label and behavior information, which can facilitate 
future research in this area.




%%% Local Variables: 
%%% mode: latex
%%% TeX-master: "paper"
%%% End: 
