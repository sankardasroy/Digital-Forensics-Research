\section{Introduction}\label{intro}

Nowadays government organizations as well as private enterprises encounter cyber-attacks quite frequently.
After such an attack law enforcement typically conducts a digital forensics (DF) investigation~\cite{df:news,cyber2011} 
with the goal of identifying what caused the attack, how it was executed, and who the potential perpetrators are. 
In addition to cyber-attacks, we encounter many other digital crimes, such as theft of intellectual properties, 
compromise of private data, and more. To conduct post-mortem analysis of cyber-attacks and digital crimes, forensics professionals
employ multiple DF tools. 

The success of the investigation depends on the accuracy and overall quality of DF tools.
Furthermore, as the investigation often culminates with a judicial proceeding in the court of law, integrity of DF tools 
is critical. On one hand, a court may throw out a piece of evidence if it was collected by a DF tool that does not follow the standards, 
and on the other hand, inaccurate results from a DF tool can cause improper prosecution of an innocent defendant. 
DF tools that are available in the market come from many vendors, and the government needs to keep an eye on the integrity of these tools. 
As a standardization initiative of DF tools,  Computer Forensics Tool Testing Program (CFTT)~\cite{cftt:nist} 
at National Institute of Standards and Technology (NIST) published a list of expectations for these tools. 

In this chapter, we consider standardization of Deleted File Recovery (DFR)~\cite{meta:dfr:standards,carving_standards} tools which is a special type of DF tools. 
Given a storage device, such as hard disk or a USB drive, a DF professional uses a DFR tool to search for and retrieve deleted files.
DFR tools find wide use in real-life DF investigations. For instance, after capturing a hard disk from a suspect's computer a
DF professional uses a DFR tool to retrieve the files that the suspect might have deleted to destroy some important artifacts.
A retrieved file might add critical evidence to the case in hand. In other words, success or failure of a DFR tool can sway the outcome of a case.

Depending on the working principle of DFR tools, they fall into two subtypes: \emph{Metadata-based DFR} tools and \emph{file carving} tools.
The first subtype utilizes the file system metadata (if available) to identify (a.k.a. recover) a deleted file. 
The second subtype (i.e., a file carving tool) does not rely on the file system metadata and instead utilizes the target file's 
header and footer signature for the recovery task. 
In this chapter, we consider both \emph{metadata-based DFR} tools and \emph{file carving} tools, and we evaluate them up to the NIST CFTT expectations.       

Scientific evaluation of a DFR tool is challenging because multiple factors of the given forensics scenario play a role in the tool's success or failure.
A few such factors are (a) whether the target file is fragmented, (b) whether the target file is (partially) overwritten by another file, 
(c) presence of other active or deleted files in the storage device (that hosts the files), etc.
When we compare DFR tools we make sure that these factors are the same to ensure a fair and scientific comparison.
To evaluate the file carving tools NIST CFTT portal provides a set of test images which are designed considering the above factors.
We use these test images in our study of file carving tools. 

For metadata-based DFR tools, in addition to the aforementioned factors, another factor is type of the file system (e.g. FAT vs. NTFS). 
Unfortunately, NIST CFTT does not provide any test image for evaluation of metadata-based DFR tools \cite{meta:dfr:standards}.
So, for each file system type we ourselves design a set of test images per NIST guidelines \cite{meta:dfr:standards}, considering each of the above factors independently.
In particular, we designed 14 test images for FAT and NTFS, and we claim that these cover most of the real-life scenarios.


Via extensive experiments we evaluated multiple frequently used DFR tools\footnote{We have chosen a few tools as the subject of our study, 
but such choosing does not imply authors' recommendation or endorsement of any particular tool.}, and we find that many tools do not meet one or more NIST expectations.
We find that a tool correctly retrieves the deleted file in some scenarios whereas it fails in other scenarios. Furthermore, we observe that 
two tools can perform differently on the same test image. From our experience, whenever possible, 
we strive to provide logical explanation of these behaviors. We find that in many cases 
success or failure of a tool depends on its design principle. 
For instance, a file carving tool may employ a \emph{strong match policy} to identify a deleted file, which can help prevent \emph{false positives} (when a tool retrieves garbage data instead of a real file),
but this may cause the tool to miss some deleted files. Conversely, employing a \emph{weak match policy} leads to opposite outcomes. 
 
We compare performance of the DFR tools and report a comparative analysis. In our opinion, such comparative analysis is beneficial to both DF tool 
users and vendors: it can help users choose the right DF tool, and it can help vendors identify room for improvement and their niche in the market. 
NIST CFTT publishes reports on DFR tools' performance, but only a subset of DFR tools are covered~\cite{cftt_meta_reports,cftt_carving_reports}. 
We need to expand the coverage as more and more tools come to the market.
Furthermore, some reports on the CFTT portal are on outdated version of the DFR tools, which warrants updating. 
For instance, at the time of writing, the reports on Autopsy~\cite{dhs:autopsy} and FTK~\cite{dhs:ftk}, two frequently used DF tools, came out in 2014. 
Both tools have received numerous updates since then.   

As we have tried to compare DFR tools' behavior in many test scenarios with NIST CFTT expectations, we have also gotten a first-hand experience of applicability 
of the NIST CFTT guidelines in a practical setting. For instance, we observe some situations where a tool may be held to an impossible standard.
We provide critique on applicability of the NIST CFTT~guidelines.


\commentOut{

As there are many file systems (e.g., ext4, HFS, etc.) in addition to FAT and NTFS, one might be interested to know why we chose FAT and NTFS for the current work. 
Because FAT and NTFS are very widely used on external storage devices and devices running Microsoft Windows, respectively,
real-life forensics investigation often involves these two file systems.
While we leave other file systems for future study, our current methodology could be 
extended to other file systems to make a similar study.


As a side benefit, our work leads to a few hands-on lab-modules to be used in digital forensics courses 
at BGSU, enriching the new DF specialization program in the CS department. We will also make these modules
publicly available to be used by relevant instructors at other universities.

} 
