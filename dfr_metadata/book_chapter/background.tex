\section{Background}

We present some of the background material in this section that will prepare
us for the latter part of the chapter. 
In particular, we present an overview of metadata-based recovery and file carving. 
Then, we present the NIST CFTT expectations for such recovery tools.  


\subsection{Metadata-based Deleted File Recovery}
This recovery mechanism relies on the residual metadata of the deleted files, 
which are still present in the file system. Note that in a typical file system 
a file has some metadata in addition to the actual file content. However,
different file systems
manage the metadata in different ways.
In Section \ref{subsubsec:fat-overview} and Section \ref{subsubsec:ntfs-overview}, 
we present an overview of FAT and NTFS file systems, respectively, which is often
simplified to aid readability. 
Then, in Section \ref{subsubsec:meta-recovery}, we discuss how metadata can help in 
recovery of a deleted~file.


\subsubsection{FAT File System} \label{subsubsec:fat-overview}

For each file, the FAT file system maintains an important piece of metadata, 
which is known as a \emph{directory entry} and is of 32 bytes, containing three 
key elements: (i) name of the file (say bar.txt), (ii) size of the file, and 
(iii) index of the starting cluster that holds the file content. 
The file system maintains the index of the other clusters of the file (bar.txt) 
via a global table known as the FAT table. For each cluster in the file system, 
there is an entry in the FAT table, which holds the status 
(allocated vs. unallocated) of the cluster, and if allocated, 
it also tells us what the next cluster is. In particular, if $x^{th}$ entry is $0$, 
we infer that the cluster is unallocated. Otherwise, if $x^{th}$ entry is $y$, then 
that means the next cluster after cluster $x$ (as part of the same file) is $y$.
When $x^{th}$ entry holds a special value called EOF, we infer that cluster $x$ is
the last cluster of a file.

For instance, Figure~\ref{fig:fat1} illustrates the directory entry and data clusters of 
a file bar.txt. The FAT table is also shown. We can infer from the FAT table that the file's
(bar.txt) content is stored in contiguous clusters, from cluster 200 to cluster 204.


\begin{figure}[h]
     \centering
     \includegraphics[width=\linewidth]{fig/fat1.pdf}
     \caption{The actual data part and metadata of file bar.txt in a FAT file system. The \emph{directory entry} of this file and the actual file content clusters (shaded) are shown on the right. The FAT table is shown on the left.}
     \label{fig:fat1}
 \end{figure}

When a file is deleted in FAT system, most of the actual content and metadata might 
remain intact in many cases. 
Figure~\ref{fig:fat2} illustrates the status of the file system after the file bar.txt is deleted.
All fields of the directory entry remain unchanged except the first character of the file name 
is replaced with an underscore (`\_') to flag the \emph{deleted} status. The main change happens in the FAT table where all entries
that were associated with bar.txt are \emph{zeroed}. However, the clusters (holding the file content) 
remain intact until some other file (partially or fully) \emph{overwrites} them. 

  
\begin{figure}[h]
    \centering
    \includegraphics[width=\linewidth]{fig/fat2.pdf}
    \caption{The actual data part and metadata of bar.txt after the file is deleted whereas its entries in the FAT table (i.e., $200$th entry to $204$th entry) are~zeroed.}
    \label{fig:fat2}
\end{figure}

So far, we have considered a file whose actual content is stored in contiguous clusters. 
It is also possible that a file's content is not stored in contiguous clusters, and such a file is 
called a \emph{fragmented} file. Figure~\ref{fig:fat3} illustrates an example where file bar.txt is fragmented.

 \begin{figure}[h]
     \centering
     \includegraphics[width=\linewidth]{fig/fat3.pdf}
     \caption{The actual content and metadata of bar.txt is shown. Per the FAT table the file has two fragments (clusters 200-201 and cluster 204).}
     \label{fig:fat3}
 \end{figure}



\subsubsection{NTFS File System} \label{subsubsec:ntfs-overview}

NTFS file system does not maintain a global table (like the FAT table) 
that holds the status for each cluster. Instead NTFS maintains a global table called 
MFT (Master File Table) that
holds information for each file in the system. In particular, each file $f$ has an 
entry in MFT, which holds information about file $f$, and the size of this entry is 1,024 bytes.  
If file $f$ is small, then the whole file content as well as the metadata will be stored
inside $f$'s MFT entry; 
otherwise, $f$'s content is \emph{non-resident} and is stored in other clusters. 
Figure~\ref{fig:ntfs} illustrates an example where the file bar.txt's content is non-resident, 
and it has two fragments. When a file $f$ is deleted in NTFS, 
the corresponding entry in MFT is flagged as deleted and the corresponding clusters 
(if any) are flagged as deleted, but the metadata and file content generally remain intact.
Note that NTFS file system does not have specific zones for data area in contrast of 
FAT having specific area for data and specific area for the FAT table.


 \begin{figure}[h]
     \centering
     \includegraphics[width=\linewidth]{fig/ntfs.pdf}
     \caption{To illustrate NTFS file system, the MFT entry of bar.txt and the actual content carrying clusters are shown. 
 This file has two fragments (clusters 200-201 and cluster 204).}
     \label{fig:ntfs}
 \end{figure}
 

\subsubsection{Recovering Deleted Files}\label{subsubsec:meta-recovery}

From the previous discussion we note that in many situations the metadata
(e.g., directory entry in FAT or MFT entry in NTFS) of the deleted file remains
unchanged and can be used for identifying and recovering the deleted file.
As an example, the directory entry of bar.txt in Figure~\ref{fig:fat2} tells us
that the file's content starts from cluster $200$, 
and using the \emph{size} field value (e.g., $5$) we can infer that the file's content
is hosted in the cluster chain from cluster $200$ to cluster $204$, assuming that there is no
fragmentation. We can recover the file by reading the raw content of these $5$ clusters, e.g.,
by using \emph{dd} command in Linux.  

We encounter one critical challenge while recovering a fragmented file in FAT because the \emph{directory entry}
of a file does not contain any information about the fragments. However, we do not face this challenge 
in NTFS file recovery because the MFT entry (as shown in Figure~\ref{fig:ntfs}) does contain the start and end cluster index for each run (i.e., fragment). 

\subsection{File Carving}\label{subsec:file-carving}

File carving does not rely on the file system metadata, and hence is independent of the type of the 
file system (e.g., FAT vs. NTFS). In file carving, we assume that the target file has 
a known header and footer signature (i.e., a sequence of special bytes). As an example, 
a JPG file has such a header and a footer signature--in particular, certain two bytes 
for the header and certain two bytes for the footer. The file carving process basically
scans the whole storage space (i.e., the target storage device where we plan to recover deleted files from) 
byte by byte and identifies each match of header and footer signature. 
Then, the content between any header and any footer is potentially a recovered file. 
Depending on the strong vs weak match policy, there could be \emph{false positives} 
(i.e., bogus files being retrieved) 
and \emph{false negatives} (i.e., files missed to be retrieved). 

\subsection{NIST CFTT Guidelines}
NIST CFTT program has published guidelines~\cite{meta:dfr:standards,carving_standards} on how to evaluate deleted file recovery (DFR) tools.

\subsubsection{For Metadata-Based DFR}
NIST's guidelines~\cite{meta:dfr:standards} for evaluating metadata-based DFR tools consist of four \emph{core features} and several \emph{optional features}. We evaluate based only on the core features and leave the optional features for later work.
In this section we list the core features as they appear in the NIST guidelines document, along with our own interpretation and commentary.

 \paragraph{DFR-CR-01} ``The tool shall identify all deleted File System-Object entries accessible in residual metadata''~\cite{meta:dfr:standards}.
 We say a tool fulfills this core feature if it reports something for each file system metadata entry which has been marked as deleted.
 
 \paragraph{DFR-CR-02} ``The tool shall construct a Recovered Object for each deleted File System-Object entry accessible in residual metadata''~\cite{meta:dfr:standards}.
 We say a tool fulfills this core feature if it outputs a file for each of the deleted files identified per DFR-CR-01, even if the output file is empty.

 \paragraph{DFR-CR-03} ``Each Recovered Object shall include all non-allocated data blocks identified in a residual metadata entry''~\cite{meta:dfr:standards}.
 Our interpretation of this feature is file system-dependent as there are differences in what information is available in metadata.
 In the FAT file system, it is impossible to detect fragmentation purely from metadata, so we say a tool fulfills this core feature if it recovers at least all unallocated clusters that were allocated to the first fragment.
 In the NTFS file system, the locations of all fragments are left in metadata after file deletion, so to fulfill this core feature, a tool must recover all unallocated clusters that were allocated to the deleted file.


 \paragraph{DFR-CR-04} ``Each Recovered Object shall consist only of data blocks from the Deleted Block Pool''~\cite{meta:dfr:standards}.
 We say a tool fulfills this core feature as long as the recovered file contains only data from the original deleted file, or null data to represent parts of the file that have been overwritten or are otherwise inaccessible.

\subsubsection{For File Carving} \label{sec:carving_features}

NIST's guidelines~\cite{carving_standards} for evaluating file carving tools consist of five \emph{core features}.
In this section we list the core features as they appear in the NIST guidelines document, along with our own interpretation and commentary.

 \paragraph{FC-CR-01} ``The tool shall return one carved file for each supported file header signature from a source file that is present in the search arena''.~\cite{carving_standards}
 Each file from the original disk will begin with a header signature specific to its file format. We say a tool fulfills this core feature if it carves a file starting at each of those header signatures.
 In other words, tools that perform well on this core feature have a high ``hit rate.''
 
 \paragraph{FC-CR-02} ``A carved file shall only contain data blocks from the search arena''.~\cite{carving_standards}
 In other words, the tool should only work within the drive or partition it is given, and should not try to carve from out-of-bounds.
 
 \paragraph{FC-CR-03} ``All data blocks in a carved file shall originate in a single source file''.~\cite{carving_standards}
 We say a tool fulfills this core feature if each recovered file only contains data from one file on the original disk.
 
 \paragraph{FC-CR-04} ``The file type of a carved file shall match the file type of its contents''.~\cite{carving_standards}
 We interpret this to mean that the file extension given to a recovered file must accurately describe the format of the file data. We exclude false positives from this evaluation because their data is highly unlikely to be of any file format. So, we only consider files which were carved starting from a valid header signature.
 
 \paragraph{FC-CR-05} ``The tool shall return carved files in a state that conforms to a valid file of the carved file type''.~\cite{carving_standards}
 We say a tool fulfills this core feature if each recovered file can be parsed without error by some application software.
 We use the ImageMagick tool suite to evaluate this.
